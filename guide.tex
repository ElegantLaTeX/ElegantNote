\documentclass[cyan]{elegantnote}

\author{ddswhu \& 小L}
\email{ddswhu@gmail.com}
\zhtitle{优美的\LaTeX{}模板}
\entitle{Elegant \LaTeX{}}
\version{1.00}
\myquote{Victory won\rq t come to us unless we go to it.}
\logo{logo.pdf}
\cover{cover.pdf}

%green color
   \definecolor{main1}{RGB}{210,168,75}
   \definecolor{seco1}{RGB}{9,80,3}
   \definecolor{thid1}{RGB}{0,175,152}
%cyan color
   \definecolor{main2}{RGB}{239,126,30}
   \definecolor{seco2}{RGB}{0,175,152}
   \definecolor{thid2}{RGB}{236,74,53}
%cyan color
   \definecolor{main3}{RGB}{127,191,51}
   \definecolor{seco3}{RGB}{0,145,215}
   \definecolor{thid3}{RGB}{180,27,131}

\usepackage{makecell}
\usepackage{lipsum}





\begin{document}
\maketitle
\tableofcontents
\chapter{Elegant Note模板的由来}

只有当自己想去做一件事的时候才能把事情做好!

\section{长长的历史,长长的期待}

写这个模板的初衷是为了简化我在写笔记中的工作,因为我不会写类文件和包文件,所以,最当初是想拜托小L做出一个简洁,清爽的\LaTeX{}模板,最好是类文件,而且因为这样可以简化导言区复杂的内容。后来,和小L一拍即合,遂开始一起做Elegant\LaTeX{}的设计。

在学校的时候,搞定了定理环境样式的代码。因为不想重复 ChinaTeX 那个经典的页眉页脚,我找到了计量书上的一个图案,小L拿 Ti\emph{k}Z 一点一点把那个画出来了,不过我最后还是用的截取的方式得到的图案。然后慢慢地,我们把初步的样子做出来了。

2013年的暑假开始后,我对那个初步的模板做了一点改动,然后用它写了Dynamic Programing 的笔记,并且,在写的过程中,对模板加了封面,也就是模板现在的封面。至此,模板的大致样子终于出来了,不过也在写笔记的过程中知道了某些不足,比如
\begin{enumerate}
\item 定理类的环境在我们这个模板中不能浮动,也不能跨页,在我们这个1.00版本中,这个功能仍然没有得到解决。
\item 某些环境不足,比如例子、假设、性质、结论等环境,在1.00版本中已经增加了这几个环境。
\item 一些我们不可预知的错误将会不期而遇。
\item 一些我们目前没有需求,但是可以继续改进的地方,比如表格样式,比如抄录样式等。
\end{enumerate}

写完那个笔记之后越发让我对Elegant\LaTeX{}模板的制作更有激情,在和小L相互讨论的几天里,我们终于得到了现在这个版本的ElegantNote模板。




\section{一张白纸折腾出一个模板}

我以前从未写过类文件,所以,写这个模板的过程必然是折腾的过程,在写模板的过程中,最主要参考了moderncv.cls文件、武汉大学黄正华老师的论文模板,以及
各大\LaTeX{}疑问解答网站。

{\color{thid}这章还有这么大空间,忍不住插个图!}

\begin{figure}[!hbtp]
\includegraphics[width=0.8\textwidth]{happy.jpg}
\caption{Happiness,We have it!\label{figur:happy}}
\end{figure}

\chapter{Elegant Note开服说明}

\section{关于字体}

首先呢?基于本模板追求视觉上的美观的角度,强烈建议使用者安装./fonts/文件夹下的字体。出于版权的考虑,务必不能将此模板用于涉及盈利目的的商业行为,否则,后果自负,本模板带的字体仅供学习使用,如果您喜欢某种字体,请自行购买正版。本文主要使用的字体如下
\begin{itemize}
\itemsep=3pt
\parskip=0pt
\item Adobe Garamond Pro
\item Minion Pro \& Myriad Pro
\item 方正字体
\item 华文中宋
\end{itemize}

并且,如果系统内安装了Adobe字体,建议大家把模板中的黑体,楷体,宋体等字替换成Adobe字体,这样可以达到最佳效果。

\begin{note}
需要特别注意的是,如果笔记需要使用到抄录环境的,请重新修改字体,此版本并未为抄录环境设置合适字体,本note环境的字体即为抄录环境使用到的字体。
\end{note}

\section{文档说明}
\subsection{编译方式}
本模板基于book文类,所以book的选项对于本模板也是有效的。但是,只支持 \XeLaTeX{},编码为 UTF-8,推荐使用 \TeX{}live编译。作者编写环境为Win8(64bit)+\TeX{}live 2013。

本文特殊选项设置共有2类,分为{\color{main}颜色}和{\color{main}数学字体}。

\subsection{选项设置}
第一类为{\color{main}颜色}主题设置,内置3组颜色主题,分别为green(default),cyan,blue。默认为green颜色主题。需要改变颜色的话请自行到elegantnote.cls文件内对颜色的RGB值进行修改。

第二类为{\color{main}数学字体}设置,有两个可选项,分别是computer modern 和 mtpro2字体,默认使用cm字体,无需在类文件前加选项,调用mtpro2字体的方法为\verb|\documentclass[mtpro]{elegantnote}|

\begin{table}[htp]
\centering
\begin{tabular}{ccccc}
\toprule	
	   & green & cyan & blue & 主要使用的环境\\ 
\midrule
main & \makecell{{\color{main1}\rule{1cm}{1cm}}}& \makecell{{\color{main2}\rule{1cm}{1cm}}}&\makecell{ {\color{main3}\rule{1cm}{1cm}}}& newdef\\

seco &\makecell{ {\color{seco1}\rule{1cm}{1cm}}}& \makecell{{\color{seco2}\rule{1cm}{1cm}}}&\makecell{ {\color{seco3}\rule{1cm}{1cm}}}&newthem \ newlemma \ newcorol\\

thid &\makecell{ {\color{thid1}\rule{1cm}{1cm}}}& \makecell{{\color{thid2}\rule{1cm}{1cm}}}&\makecell{ {\color{thid3}\rule{1cm}{1cm}}}&newprop\\
\bottomrule
\end{tabular}
\caption{Elegant note 模板中的三套颜色主题\label{tab:color thm}}
\end{table}

\subsection{数学环境简介}
在我们这个模板中,定义了三大类环境
\begin{enumerate}
\item 定理类环境,包含标题和内容两部分。根据格式的不同分为3种
\begin{itemize}
\item {\color{main} newdef} 环境,含有一个可选项,编号以章节为单位;
\item {\color{main}newthem、newlemma、newcorol} 环境,三者颜色一致,但是定理环境编号以章节为单位,引理和推论为全文编号;
\item newprop 环境,含有可选项,编号以章节为单位。
\end{itemize}
\item 证明类环境,有{\color{main}newproof、note} 环境,特点是,有引导符和引导词,并且证明环境有结束标志。
\item 示例环境,有{\color{main} example、assumption、conclusion} 环境,三者均以粗体的引导词为开头,字体以灰色,和普通段落格式一致。
\end{enumerate}

\subsection{可编辑的字段}
在模板中,可以编辑的字段分别为作者\verb|\author|、\verb|\email|、\verb|\zhtitle|、\verb|\entitle|、\verb|\version|。并且,可以根据自己的喜好把封面水印效果的\verb|cover.pdf|替换掉,以及封面中用到的\verb|logo.pdf|。

\chapter{笔记写作示例}

\section{灵魂不随便出卖,代码也不随便瞎写}
\lipsum[3]
考虑如下的随机动态规划问题
\begin{align*}
&\max(\min)\quad \mathbb{E}\int_{t_0}^{t_1}f(t,x,u)\,dt\\
&\quad\mbox{s.t.} \quad dx=g(t,x,u)dt+\sigma(t,x,u)dz\\
&\quad \hspace{2.em} k(0)=k_0\;\text{given}
\end{align*}

where $z$ is stochastic process or white noise or wiener process.

\begin{newdef}[Wiener Process]
If $z$ is wiener process, then for any partition $t_0,t_1,t_2,\ldots$ of time interval, the random variables $z(t_1)-z(t_0),z(t_2)-z(t_1),\ldots$ are independently and normally distributed with zero means and variance $t_1-t_0,t_2-t_1,\ldots$
\end{newdef}

\lipsum[1-2]

\begin{newthem}[勾股定理]
勾股定理的数学表达为
\[a^2+b^2=c^2\]
其中$a,b$为直角三角形的两条直角边长,$c$为直角三角形斜边长。
\end{newthem}

\begin{note}
因为引理,推论的样式和定理的样式一致,仅仅只有计数器的设置不一样,在这里,我们就不写引理和推论的例子了。
\end{note}


\lipsum[4]

\begin{newprop}[最优性原理]
如果$u^*$在$[s,T]$上为最优解,则$u^*$在$[s,T]$任意子区间都是最优解,假设区间为$[t_0,t_1]$的最优解为$u^*$,则$u(t_0)=u^{*}(t_0)$,即初始条件必须还是在$u^*$上。
\end{newprop}

\lipsum[5-6]
\begin{newcorol}
假设$V(\cdot,\cdot)$为值函数,则跟据最大值原理,有如下推论
\[
V(k,z)=\max\Big\{u\big(zf(k)-y\big)+\beta \mathbb{E}V(y,z^\prime)\Big\}
\]
\end{newcorol}

\begin{newproof}
因为 $y^*=\alpha\beta z k^\alpha$,$V(k,z)=\alpha/1-\alpha\beta\ln k_0+1/1-\alpha\beta \ln z_0+\Delta$。
\begin{align*}
\text{右边}&=\Big\{u\big(zf(k)-y\big)+\beta \mathbb{E}V(y,z^\prime)\Big\}\\
&=\ln(zk^\alpha-\alpha\beta zk^\alpha)+\beta\mathbb{E}\Big[\frac{\alpha}{1-\alpha\beta}\ln y+\frac{1}{1-\alpha\beta}\ln z^\prime+\Delta\Big]\\
&=\ln(1-\alpha\beta)zk^\alpha+\beta\Big\{\mathbb{E}\big[\frac{\alpha}{1-\alpha\beta}\ln \alpha\beta z k^\alpha\big]+\frac{1}{1-\alpha\beta}\mathbb{E}[\ln z^\prime]+\Delta\Big\}
\end{align*}
利用$\mathbb{E}[\ln z^\prime]=0$,并将对数展开得
\begin{align*}
\text{右边}&=\ln (1-\alpha\beta)+\ln z+\alpha\ln k+\frac{\alpha\beta}{1-\alpha\beta}\big[\ln \alpha\beta+\ln z+\alpha\ln k\big]+\frac{\beta}{1-\alpha\beta}\mu+\beta \Delta\\
&=\frac{\alpha}{1-\alpha\beta}\ln k+\frac{1}{1-\alpha\beta}\ln z+\Delta
\end{align*}
所以$\text{左边}=\text{右边}$,证毕。
\end{newproof}




\begin{conclusion}
今天看到一则小幽默,是这样说的:{\color{main} 别人都关心你飞的有多高,只有我关心你的翅膀好不好吃!}说多了都是泪啊!
\end{conclusion}

最后祝大家\LaTeX{}的学习之路快乐精彩!

\end{document}
